% !TEX = root../thesis.tex

\chapter{Experimental Apparatus}
\label{chap:exp}

\section{The Large Hadron Collider} \label{sec:LHC}
The Large Hadron Collider (LHC) is a circular collider spanning the border between France and Switzerland, based at the European Organization for Nuclear Research (CERN). The central feature of the LHC is the superconducting rings, located about 100 m underground with a circumference of 27 km, designed to collide counter-rotating beams of protons or heavy ions at highly relativistic energies. Along the rings lie four major experiments: ATLAS, CMS, ALICE, and LHCb. Both ALTAS and CMS are general purpose detectors, designed to probe a wide range of physics including the Higgs boson, precision measurements, and physics beyond the standard model (BSM). The remaining two experiments are more specialized; ALICE measures quark gluon plasma produced in heavy ion collisions and LHCb focuses on $b$-quark physics and CP violation.
Two prominent aspects of the LHC are the high center of mass energy of 13 TeV, referred to using the Mandelstam variable $\sqrt{s}$, and high instantaneous luminosity $\lumi$, often referred to as just luminosity. High $\sqrt{s}$ allows for the production of heavier particles, including those beyond the standard model, while high luminosity is essential for measuring rare processes and precision measurements. A process with cross section $\sigma$ will have a rate $R$ given by
\begin{equation}
	R=\lumi\sigma
\end{equation}
In cases where the relevant quantity is the total number of events, the integrated luminosity can be defined as $\intlumi=\int{\lumi\dd{t}}$ to give
\begin{equation}
	N_\mathrm{events} = \intlumi\sigma
\end{equation}