% !TEX = root../thesis.tex

\chapter{Introduction}
\label{chap:intro}
Understanding the composition of matter and the forces that govern it are the core principles motivating particle physics. The idea that all matter is composed of some fundamental building block dates back to 440 B.C. in ancient Greece, where these blocks were referred to as \textit{atomos}, meaning ``indivisible''~\cite{wolfgang2019}. This is the origin of the modern \textit{atom}, coined in 1803 by John Dalton. In his atomic theory, Dalton proposed that all matter was composed of tiny, individual particles called atoms which could neither be created nor destroyed~\cite{oed_physics}. This theory held until 1897, when J.J. Thomson discovered that cathode rays emitted from a hot filament were actually beams of charged particles~\cite{thomson_electron}. Thomson theorized that atoms were not fundamental but composite particles, consisting of a positively charged ``paste'' with his charged particles, later named electrons, embedded within. Thomson's model of the atom was further refined in 1911, when Ernest Rutherford's gold foil experiment revealed that atoms consisted of a dense, charged core and resulted in the discovery of the proton~\cite{rutherford}. Finally, in 1932, the classical model of the atom was completed when James Chadwick discovered the neutron by observing radiation emitted from a beryllium target that was bombarded with alpha particles~\cite{chadwick}.

In parallel to classical atomic theory, quantum theory began to develop in 1900 when Max Planck solved the ultraviolet catastrophe -- a problem in which the power radiated from an ideal blackbody was calculated to be infinite -- by assuming that energy from electromagnetic radiation was quantized~\cite{kittel1980thermal}. This energy was revealed to be discrete particles with quantized energies following the work of Albert Einstein, through his explanation of the photoelectric effect in 1905, and A. H. Compton in 1923, through his discovery of Compton scattering. These particles, later named \textit{photons}, were the first interpretation of particles acting as mediators of a force.

The next question for particle physics was to determine the mechanism that held the nucleus together. Physicists were certain it must be a new force, as the protons would repel each other through the electromagnetic force and were much too light for gravity to play any significant role. In 1934, Yukawa proposed the first theory of the strong force and predicted a mediator particle with mass about one-sixth of the proton mass~\cite{QFTNutshell}. He called this intermediate particle a \textit{meson} -- a middle ground in mass between the light, electron scale particles (\textit{leptons}) and heavy, proton scale particles (\textit{baryons}). The search for this strong mediator eventually led to the discovery of both the pion (\PGp) and muon (\PGm)~\cite{pion_discovery}, with the pion being the meson Yukawa predicted and the muon being a lepton from the pion decay -- though neither were the true mediator of the strong force.

Throughout the next decade a myriad of new particles were discovered that verified theoretical predictions, such as the positron (and more generally antiparticles) in 1932 by C. D. Anderson~\cite{positron_discovery} and the neutrino in the 1950s by Cowan and Reines\cite{neutrino_discovery}. Additionally, several ``strange'' mesons and baryons were discovered using cloud chambers and particle accelerators, such as the kaon (\PKz) in 1947 by Rochester and Butler~\cite{kaon_discovery} and the lambda (\PGL) in 1950 by Hopper and Biswas~\cite{lambda_discovery}. In order to develop a unified model to accommodate all of the newly discovered particles, Gell-Mann and Nishijima independently devised the property of particle \textit{strangeness}~\cite{strangeness_gellmann,strangeness_nishijima}, eventually leading to Gell-Mann creating the Eightfold Way in 1961, which mapped baryons and mesons on an elegant, if not convoluted, grid of strangeness and electric charge~\cite{eightfoldway}. This grid had one notable gap, which Gell-Mann predicted would be a new particle and resulted in the discovery of the omega-minus (\PGOm) in 1964~\cite{omega_discovery}.

In 1964 Gell-Mann and Zweig independently proposed that baryons and mesons were not actually fundamental particles, but were instead composed of elementary particles called \textit{quarks}~\cite{gellmann_quarks,zweig_quarks}. Once again what scientists believed were the smallest constituents of matter were broken down into even smaller units. In this theory, there are three quarks (\PQu, \PQd, and \PQs) that each carry a charge and strangeness, every meson is composed of a quark-antiquark pair, and every (anti)baryon is composed of three (anti)quarks. From this, the complex grid in Gell-Mann's Eightfold Way followed naturally by mapping every allowed combination of quarks. The first evidence supporting the quark theory came in 1969 from the Stanford Linear Accelerator Center (SLAC), in an experiment analogous to Rutherford's gold foil experiment, by scattering electrons off of protons and measuring the deflection~\cite{slac_quark}. Further evidence came in the form of the \PJGy meson discovery in November, 1974 from Brookhaven~\cite{jpsi_brookhaven} and SLAC~\cite{jpsi_slac}, which indicated the existence of a fourth flavor of quark -- the charm (\PQc).

The discovery of the \PJGy meson spurred what is known as the November revolution, a period of rapid discovery in the world of particle physics~\cite{revolution}. As expected, several baryons and mesons containing the charm quark were discovered, but the periodic table of fundamental particles gained several new entries as well. In 1975, the tau lepton (\PGt) was discovered along with its corresponding neutrino~\cite{tau_discovery}, and in 1977 the discovery of the upsilon meson (\PGU) indicated the existence of a fifth flavor of quark, the bottom (\PQb)~\cite{upsilon_discovery}. At this point, there were six leptons and five quarks, with the existence of a sixth quark predicted to maintain parity between the quarks and leptons. The final quark, dubbed the top (\PQt), was eventually discovered in 1995 at the Tevatron~\cite{top_discovery}, and is the last quark discovered as of present day.

The theory that describes the fundamental particles and the interaction forces is known as the Standard Model (SM). It perfectly describes the behavior of the quarks and leptons, though this is by construction as most of the parameters were determined empirically through experiment. For this reason, the predictive power of the SM was not truly validated until the discovery of the \PW, \PZ, and (most importantly) Higgs bosons. The first two are the force mediators of the weak force, which governs nuclear decay, and were predicted by Glashow, Weinberg, and Salem from their work on electroweak unification~\cite{ew_nobelprize} before being discovered in 1983 at the European Organization for Nuclear Research (CERN)~\cite{w_discovery,z_discovery}. The Higgs boson, which was predicted to give mass to the fundamental particles, managed to evade detection until the Large Hadron Collider began taking data in 2010. After two years of data taking the Higgs boson was finally discovered in 2012~\cite{higgs_atlas,higgs_cms}, marking a major milestone for the predictive power of the SM.

The discovery of the Higgs boson was the last missing piece of the Standard Model, but the puzzle of fundamental particles is still incomplete. Dark matter, which comprises 27\% of mass-energy of the observable universe, is not predicted by any SM particle. Moreover, there is no evidence for the existence of stable antimatter in the universe and no mechanism in the Standard Model that fully explains this matter-antimatter asymmetry. If gravity is to be incorporated into the SM, then it must be quantized, similar to Einstein and his quantization of light, but gravity and general relativity are thought to be incompatible with quantum theory~\cite{sm_grav}. Neutrinos, which are assumed by the SM to be massless, have experimentally been shown to have non-zero mass. Theories seeking to resolve these issues are referred to as Beyond the Standard Model (BSM), and are the motivation for several ongoing experiments at the LHC.

This thesis describes a search for Higgs boson decays to long-lived, BSM bosons ($\Phi$) known as axion-like particles (ALPs) which then decay to two photons. These ALPs are predicted in several BSM theories, hence the analysis is intended to be independent of the specific theoretical model to provide maximum discovery potential. The data used in this analysis correspond to proton-proton collisions with center-of-mass energy of 13\TeV collected by the Compact Muon Solenoid (CMS) detector at the LHC from 2016-2018. In order to provide the full context for the analysis, chapter~\ref{chap:theory} will give an overview of the Standard Model and the theoretical motivation for ALPs. Chapter~\ref{chap:exp} then describes the LHC and CMS detector used to obtain the data for this analysis, with chapter~\ref{chap:kbmtf} detailing work done for the real time muon reconstruction done by the level-1 trigger in CMS. The full $\PH\to\Phi\Phi\to\PGg\PGg+X$ analysis itself is then given in chapter~\ref{chap:ana}.