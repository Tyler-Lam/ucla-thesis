% !TEX = root../thesis.tex

\chapter{The Standard Model and Motivations for Displaced Photons}
\label{chap:theory}
In this chapter, we present an overview of the Standard Model (SM) -- the best current description of the fundamental constituents of matter and the interactions between them. We begin with a brief description of the SM particles, followed by an outline of the mechanism that determines particle lifetimes. This establishes the groundwork to discuss the Higgs boson its relation to long-lived, beyond the standard model particles that motivate the $\PH\to\Phi\Phi\to\PGg\PGg+X$ analysis presented in chapter~\ref{chap:ana}.

\section{Elementary Particles in the Standard Model} \label{sec:SM}
The Standard Model is a quantum field theory that describes interactions between fundamental particles (see appendix~\ref{sec:sm_theory_qft} for a description of the mathematical formalism). According to the SM, all matter is composed of spin-1/2 particles known as fermions with forces that interact via fields mediated by spin-1 gauge bosons. Of the four forces -- gravity, electromagnetic, weak, and strong -- the SM provides a description of all but gravity (see appendix~\ref{sec:sm_theory_gauge}). The spin-1/2 fermions can be divided into three generations of increasing mass, with each generation divided into a quark doublet and lepton doublet. The gauge bosons are composed of the photon ($\PGg$) which mediates the electromagnetic force, the \PZ and \PWpm bosons which mediate the weak force, and 8 gluons (\Pg) which mediate the strong force. The \PZ and \PW bosons gain mass through a process known as the Higgs mechanism, which predicts the existence of a scalar boson \PH known as the Higgs boson. This process is presented in detail in section~\ref{sec:sm_theory_higgs}. Figure~\ref{tab:SM} shows the mass, charge, and spin of the fermions grouped by generation, the gauge bosons, and the Higgs boson. Each fermion has an associated antiparticle with identical spin and mass but opposite electrical charge.

\begin{figure}[htb!]
	\centering
	% !TEX = root../../thesis.tex

% Colors from https://latexcolor.com/
\definecolor{aqua}{rgb}{0.5, 1.0, 1.0}
\definecolor{spring}{rgb}{0..65, .99, 0.0}
\definecolor{brick}{rgb}{.8, .25, .33}
\definecolor{ube}{rgb}{0.82, 0.62, 0.91}

\begin{tikzpicture}
	\coordinate (u) at (0,0);
	\coordinate (d) at (0, -2.15);
	\coordinate (c) at (2.15, 0);
	\coordinate (s) at (2.15, -2.15);
	\coordinate (t) at (4.3, 0);
	\coordinate (b) at (4.3, -2.15);
	\coordinate (e) at (0, -4.3);
	\coordinate (mu) at (2.15, -4.3);
	\coordinate (tau) at (4.3, -4.3);
	\coordinate (ne) at (0, -6.45);
	\coordinate (nmu) at (2.15, -6.45);
	\coordinate (ntau) at (4.3, -6.45);
	\coordinate (glu) at (6.75, 0);
	\coordinate (g) at (6.75, -2.15);
	\coordinate (z) at (6.75, -4.3);
	\coordinate (w) at (6.75, -6.45);
	\coordinate (h) at (9.20, 0);
	
	% quarks
	\draw pic at (u) {particle={aqua!50}{$\PQu$}{up}{$2.16\MeVcc$}{$2/3$}{$1/2$}}; %Up quark
	\draw pic at (d) {particle={aqua!50}{$\PQd$}{down}{$4.70\MeVcc$}{$-1/3$}{$1/2$}}; % Down quark
	\draw pic at (c) {particle={aqua!50}{$\PQc$}{charm}{$1.27\GeVcc$}{$2/3$}{$1/2$}}; %Charm
	\draw pic at (s) {particle={aqua!50}{$\PQs$}{strange}{$93.5\MeVcc$}{$-1/3$}{$1/2$}}; %Strange
	\draw pic at (t) {particle={aqua!50}{$\PQt$}{top}{$172.57\GeVcc$}{$2/3$}{$1/2$}}; %Top
	\draw pic at (b) {particle={aqua!50}{$\PQb$}{bottom}{$4.183\GeVcc$}{$-1/3$}{$1/2$}}; %Bot
	
	% leptons
	\draw pic at (e) {particle={spring!35}{$\Pe$}{electron}{$511\keVcc$}{$-1$}{$1/2$}};
	\draw pic at (mu) {particle={spring!35}{$\PGm$}{muon}{$105.66\MeVcc$}{$-1$}{$1/2$}}; 
	\draw pic at (tau) {particle={spring!35}{$\PGt$}{tau}{$1.776\GeVcc$}{$-1$}{$1/2$}}; 
	\draw pic at (ne) {particle={spring!35}{$\PGne$}{\Pe neutrino}{$<1.1\eVcc$}{$0$}{$1/2$}};
	\draw pic at (nmu) {particle={spring!35}{$\PGnGm$}{\PGm neutrino}{$<0.17\MeVcc$}{$0$}{$1/2$}};
	\draw pic at (ntau) {particle={spring!35}{$\PGnGt$}{\PGt neutrino}{$<18\MeVcc$}{$0$}{$1/2$}};	
	
	% gauge bosons
	\draw pic at (glu) {particle={brick!30}{$\Pg$}{gluon}{$0$}{$0$}{$1$}};
	\draw pic at (g) {particle={brick!30}{$\PGg$}{photon}{$0$}{$0$}{$1$}};
	\draw pic at (z) {particle={brick!30}{$\PZ$}{Z boson}{$91.188\GeVcc$}{$0$}{$1$}};
	\draw pic at (w) {particle={brick!30}{$\PW$}{W boson}{$80.37\GeVcc$}{$\pm1$}{$1$}};
	
	% Higgs
	\draw pic at (h) {particle={ube!50}{$\PH$}{Higgs boson}{$125.2\GeVcc$}{0}{0}};
	
	% Labels
	\node at ($(u)+(-1, 0.8)$) [anchor=mid east, scale=0.6] {mass};
	\node at ($(u)+(-1, 0.5)$) [anchor=mid east, scale=0.6] {charge};
	\node at ($(u)+(-1, 0.2)$) [anchor=mid east, scale=0.6] {spin};
	
	\node at ($(u)+(0, 1.35)$) {I};
	\node at ($(c)+(0, 1.35)$) {II};
	\node at ($(t)+(0, 1.35)$) {III};	
	%\node at ($(c)+(0, 1.9)$) {Generations};
	\node at ($(u)+(-1,1.35)$) [anchor=east, scale=0.75] {Generation};
	
	\node at ($(d)+(-1.35, -1)$) [rotate=90,anchor=mid west, text=aqua!60!black] {Quarks};
	\node at ($(ne)+(-1.35, -1)$) [rotate=90,anchor=mid west, text=spring!60!black] {Leptons};
	\node at ($(w)+(1.35, -1)$) [rotate=90,anchor=mid west, text=brick!60!black] {Gauge Bosons};
\end{tikzpicture}

	\caption[Table of all Standard Model particles, divided by generation and particle type. Each particle is labeled with its mass, charge, spin, and symbol. Quarks are shown in blue, leptons in green, gauge bosons in red, and the Higgs boson in purple.]{Table of all Standard Model particles, divided by generation and particle type. Each particle is labeled with its mass, charge, spin, and symbol. Quarks are shown in blue, leptons in green, gauge bosons in red, and the Higgs boson in purple. Values taken from~\cite{Workman:2022ynf}.}
	\label{tab:SM}
\end{figure}

\subsection{Fermions} \label{sec:sm_quarks}
Quarks are fermions that interact with the strong, weak, and electromagnetic force. They are found exclusively in bound states known as hadrons. The two main types of hadrons are quark-antiquark pairs called mesons or three quark (or three antiquark) configurations known as baryons. For example, protons, the most commonly known baryons, are composed of two up quarks and one down quark (\PQuns\PQuns\PQdns), and the \PGpp is a meson composed of an up quark and a down antiquark (\PQuns\PAQdns). Each column of quarks in table~\ref{tab:SM} forms a doublet, with the upper quark having charge $+2/3$ while the lower quark has charge $-1/3$. Thus mesons can have charge 0 or $\pm$1 and baryons can have charge 0, $\pm$1, or $\pm$2.

Due to their interaction with the strong force, quarks carry one of three ``color'' charges of either red, blue, or green (or antired, antiblue, or antigreen). Although it is not theoretically required, it has been experimentally shown that quarks must always exist in a colorless bound state. Baryons must have one quark of each color (or anti-color), and mesons must consist of color-anticolor pairs. This property is known as color confinement.

Leptons are fermions consisting of three charged particles -- the electron, muon, and tau (\Pe, \PGm, and \PGt) -- and three corresponding neutral particles (\PGne, \PGnGm, and \PGnGt) called neutrinos. The charged leptons interact through both the electromagnetic and weak force, while the neutrinos interact only through the weak force. The lepton and neutrino doublets are intrinsically linked through the weak force. Neutrinos have extremely small masses compared to the other massive SM particles, and can undergo oscillations between flavors.

\subsection{Bosons} \label{sec:sm_bosons}
The gauge bosons are the spin-1 mediators of the electromagnetic, weak, and strong forces. The photon is the massless mediator of the electromagnetic force and couples to charged particles. Since they do not carry electric charge, photons do not directly interact with other photons. On the other hand, gluons, which are the massless mediators for the strong force, carry two color charges and can therefore interact with other gluons. Lastly, the \PZ and \PWpm bosons are the mediators of the weak force and can interact with all fermions as well as each other.

The Higgs boson is the only currently known scalar boson. In the SM, the \PZ and \PWpm bosons as well as all fermions are required to be massless due to various symmetry requirements. The Higgs mechanism, through a process known as spontaneous symmetry breaking, allows these particles to acquire non-zero mass without violating conservation laws. In this sense, the Higgs boson is said to give masses to the particles. The Higgs boson couples to any particle with mass, which includes itself, the aforementioned \PZ and \PWpm bosons, and all SM fermions.

\section{Fermi's Golden Rule and Particle Lifetimes} \label{sec:theory_fermi}
One of the key results from QFT relates the interactions between particles to the rate at which they decay. Fermi's Golden Rule, which is more commonly known for describing transition rates using perturbation theory in QM, can be applied to Lorentz invariant particle decays~\cite{pdg2024}. For the decay $1\to2+3+...+n$, the partial decay rate is given by
\begin{equation}
	\dd{\Gamma}=\frac{(2\pi)^4}{2m_1}|\mathcal{M}|^2\dd{\Phi_n(p_1,p_2,...,p_n)}
\end{equation}
where $\mathcal{M}$ is the matrix element of the decay determined by the coupling constant and the interaction term of the initial and final states in the Lagrangian and $\dd{\Phi_n(p_1,p_2,...,p_n)}$ is a differentiable element of the phase space given by
\begin{equation}
	\dd{\Phi_n(p_1,p_2,...,p_n)}=\delta^4(p_1-\sum_{i=2}^n p_i)\prod_{i=2}^n\frac{\dd[3]{p_i}}{(2\pi)^32E_i}
\end{equation}
For a two body decay, integrating over the phase space gives a total decay rate of
\begin{equation}
	\Gamma=\frac{|p_i|}{8\pi m_1^2}|\mathcal{M}|^2
\end{equation}
This is the decay rate for the specific decay mode $1\to2+3$. One parameter of interest is the total decay rate for all processes, given by $\Gamma=\sum_{i}\Gamma_i$, where $i$ sums over all possible decay modes. Particles decays follow a Poisson distribution, meaning that the decay probability is independent of how long the particle has existed. The probability for a particle with total decay rate $\Gamma$ to decay after a time $t$ is given by
\begin{equation}
	\mathcal{P}(t\,|\,\Gamma)=\Gamma e^{-\Gamma t}=\frac{1}{\tau}e^{-t/\tau}
\end{equation}
where the mean lifetime $\tau=1/\Gamma$.

For a two body decay, the invariant mass distribution of the decay products forms a peak around the invariant mass of the parent particle, which is often referred to as a \textit{resonance}. The shape of this distribution is known as a Breit-Wigner distribution and is related to the decay rate. Suppose the total Hamiltonian of the system is $H=H_0+H'$, where $H_0$ has energy eigenstates and eigenvalues given by $H_0\ket{\psi_n}=E_n\ket{\psi_n}$ and $H'$ is treated as a small perturbation. Using first order perturbation theory, if the initial particle begins in state $\psi_i$, the probability to measure the parent particle in a state with energy $E_j$ is given by
\begin{equation}
	\left|c_j(t)\right|^2=\frac{\left|\bra{\psi_j}H\ket{\psi_i}\right|^2}{(E_j-E_i)^2+\Gamma^2/4}
\end{equation}
This curve has a peak at $E_i$ and a full width at half maximum (FWHM) equal to $\Gamma$. As a consequence, the longer a particle's lifetime the broader the peak in its mass distribution. Although it appears the energy of the system changed after the decay, conservation of energy is not violated as the initial state $\ket{\psi_i}$ is not an eigenstate of the full Hamiltonian and does not have a well defined energy~\cite{breitwigner}. Particles can have long lifetimes for several factors, such as phase space restrictions or suppressed decays in the matrix element.

\section{Motivations for Long Lived Scalar Bosons} \label{sec:BSM}
Despite the success of the Standard Model as the most accurate and precise description of fundamental particles and their interactions, there exist substantial shortcomings that remain unanswered. Section~\ref{sec:theory_motivation} will outline a few of the outstanding problems with the Standard Model, with potential solutions involving long-lived axion-like particles presented in section~\ref{sec:LLPs}.

\subsection{Beyond the Standard Model} \label{sec:theory_motivation}
There exists numerous observed and theoretical phenomena that motivates the need for physics beyond the Standard Model. Dark matter, which is required to explain the rotational velocity of distant galaxies~\cite{zwicky_dm}, must be a massive, neutral, stable particle -- which currently does not exist in the Standard Model. Another notable absence, as mentioned in section~\ref{sec:SM}, is a description of the gravitational force. Additionally, the derivation of the electroweak unification in section~\ref{sec:sm_theory_ew} assumed neutrinos were massless (and thus only exist as left-handed fermions), but experimental observations of neutrino flavor oscillations indicate that the neutrinos must have non-zero mass. Even the exact nature of the Higgs potential is unknown, as the potential used in~\ref{sec:sm_theory_beh} is not unique and merely the simplest form that produces three massive and one massless gauge bosons. This section will cover a few of the most relevant issues with the Standard Model as they pertain to this dissertation.

\subsubsection{Baryon Asymmetry} \label{sec:baryon_asymmetry}
It is assumed that matter and antimatter were originally created in equal quantities, yet all the baryons in the universe are composed of matter instead of antimatter. The critical phenomena to create this asymmetry are baryon number violation, periods of thermal fluctuation, and CP violation, meaning the rate for a given process is different from the rate of the CP-conjugate process~\cite{Sakharov:1967dj}. The standard model does contain the necessary mechanisms for CP violation through phase factors in the mixing matrices between flavor and mass eigenstates, as demonstrated by Kaon regeneration and neutrino oscillations. However, the amount of CP violation allowed by the Standard Model only accounts for a minuscule fraction of this asymmetry while the rest remains unexplained~\cite{Peskin2002}. One leading theory is Electroweak Baryogenesis (EWBG), a process where a first-order electroweak phase transition occurs as the early universe cools to $T<100\GeV$ which results in an excess of baryonic matter~\cite{ewbg1,ewbg2}.

\subsubsection{The Hierarchy Problem} \label{sec:hierarchy}
The energy at which gravitational effects become dominant for particle physics experiments is known as the Planck mass and can be calculated using fundamental constants as
\begin{equation}
	m_\text{Pl}=\sqrt{\frac{\hbar c}{G}}\approx10^{19}\GeVcc
\end{equation}
While this value does not contradict the Standard Model, it indicates experiments that could probe this energy scale are well out of reach for modern particle accelerators. It also creates a hierarchy problem regarding the mass of the Higgs boson that arises when accounting for the corrections to the Higgs boson mass through renormalization. Using loop corrections, one must choose a momentum cutoff $\Lambda_\text{UV}$ to evaluate the integrals that contribute to the mass of the Higgs boson~\cite{Peskin:1995ev}. The ``natural'' choice, if the Standard Model is believed to be valid for all mass scales, would be to take $\Lambda_\text{UV}\to m_\text{Pl}$. However, these corrections go as $O(\Lambda_\text{UV}^2)$, causing the calculated Higgs boson mass to quickly grow past the observed value.

To prevent these divergences requires the parameters of the standard model to be manipulated (or ``fine-tuned'') in order to produce the observed Higgs boson mass. This fine-tuning is allowed by the Standard Model, as the Higgs vacuum expectation value is a free parameter, but there are several theoretical models that remove the need for fine-tuning entirely~\cite{hierarchy}. Supersymmetry, for example, theorizes every fermion has a supersymmetric bosonic counterpart (and every boson a fermionic counterpart) and results in the loop contributions perfectly canceling term by term~\cite{susy}.

\subsubsection{Muon Anomalous Magnetic Moment} \label{sec:gminus2}
The Dirac equation predicts the muon to have an anomalous magnetic moment given by
\begin{equation}
	\mathbf{M}=g_\mu\frac{e}{2m_\mu}\mathbf{S}
\end{equation}
where $\mathbf{S}$ is the spin angular momentum, $e$ is the elementary charge, $m_\mu$ is the mass of the muon, and $g_\mu$ is the muon g-factor. By using the tree level interaction from the QED Lagrangian, one can calculate that the first order value of $g_\mu=2$~\cite{gminus2_slides}. However, higher order interactions shown in figure~\ref{fig:muon_magnetic_moment} contribute very small corrections to the g-factor, causing the theoretical value of $g_\mu$ to be very slightly more than 2.
\begin{figure}[htb!]
	\begingroup
	\tikzset{every picture/.style={scale=0.5}}
	\input{figs/02_theory/gminus2_qed.tex}
	\input{figs/02_theory/gminus2_ew.tex}
	% !TEX = root = ../../thesis.tex
\begin{tikzpicture}
	\begin{feynman}
		\coordinate[label=left:$\PGm$] (mu1) at (0,0);
		\coordinate[label=right:$\PGm$] (mu2) at (8,0);
		\coordinate (v1) at (2,1);
		\coordinate (v2) at (6,1);
		\coordinate (v3) at (4,{1+2*sqrt(3)});
		\coordinate (v4) at (3.25,1);
		\coordinate (v5) at (4.75,1);
		\coordinate[label=above:$\PGg$] (g1) at (4,6);
		\draw[fermion] (mu1) -- (v1);
		\draw[fermion] (v1) -- (v3);
		\draw[fermion] (v3) -- (v2);
		\draw[fermion] (v2) -- (mu2);
		\draw[photon] (v3) -- (g1);
		\draw[photon] (v1) -- (v4);
		\draw[photon] (v5) -- (v2);
		
		\draw[fermion] (4.75,1) arc (0:180:0.75);
		\draw[fermion] (3.25,1) arc (180:360:0.75);
		\node[label={left:$\PGm$}] at (3,{1+sqrt(3)}){};
		\node[label={right:$\PGm$}] at (5, {1+sqrt(3)}){};
		\node[label={below:$\PGg$}] at (2.625, 1){};
		\node[label={below:$\PGg$}] at (5.375,1){};
		\node[label={below:$\PQq$}] at (4, .25){};
		\node[label={above:$\PAQq$}] at (4, 1.75){};
	\end{feynman}
\end{tikzpicture}
	\endgroup
	\caption[First-order QED and lowest-order weak/hadronic contributions to the muon anomalous magnetic moment.]{First-order QED and lowest-order weak/hadronic contributions to the muon anomalous magnetic moment.}
	\label{fig:muon_magnetic_moment}
\end{figure}
The deviation of $g_\mu$ from 2 is parameterized using the magnetic anomaly defined as
\begin{equation}
	a_\mu\equiv\frac{g_\mu-2}{2}
\end{equation}
Calculating the magnetic anomaly through 5 loops in QED corrections, 2 loops in electroweak corrections, and 3 loops in hadronic corrections gives $a_\mu^\text{SM}=116\,591\,810(1)(40)(18)\times10^{-11}$, where the uncertainties correspond to the electroweak, lowest-order hadronic, and higher-order hadronic corrections~\cite{pdg2024}. Following the results on the Fermilab Muon $g-2$ experiment, which was the most precise measurement of the magnetic anomaly to date, the best measurements place the value of the magnetic anomaly at $a_\mu^\text{exp}=116\,592\,059(22)\times10^{-11}$~\cite{gminus2}. This represents a difference of $\Delta a_\mu=249(22)(43)\times10^{-11}$ between the measured and predicted values, which is a discrepancy of $5.2\sigma$~\cite{pdg2024}. While it is worth noting that recent lattice QCD calculations for the leading-order hadronic corrections reduces the discrepancy to $1.5\sigma$, this introduces additional tension between the data-driven and lattice QCD predictions of $2.2\sigma$~\cite{gminus2_latticeqcd}. In either case, these results could indicate that interactions with BSM particles are contributing to the corrections to $g_\mu$.

\subsection{Long-Lived Particles} \label{sec:LLPs}
There are ongoing experimental analyses at the Large Hadron Collider (LHC) as part of a comprehensive physics program to search for signs of BSM physics. As of yet, there has been no significant evidence of new physics; however, if BSM particles exist and have non-zero mass, then they would likely couple to the Higgs boson, meaning that they could be produced in rare decays of the Higgs boson. Rare Higgs boson decays are largely unconstrained sources of potential BSM physics, as the current limits on the branching ratio to invisible decay products is still $<10.7\%$~\cite{pdg2024}.

Several theoretical models predict BSM scalar bosons with proper lifetimes $c\tau>1\unit{mm}$ and masses on the order of $10\GeV$, which is distinct from any known Standard Model particle. One such particle arises from singlet scalar extensions to the standard model resulting in Axion-like particles (ALPs), a class of light (pseudo) scalar bosons that couple to the Higgs boson~\cite{atlas_alp}. The most general ALP model assumes a spin-0 gauge singlet $a$ with an effective Lagrangian given by~\cite{alp_colliders}
\begin{align}
	\label{eq:alp_lagrangian}
	\begin{split}
		\mathcal{L}_\text{eff}=&\frac{1}{2}\left(\partial_\mu a\right)\left(\partial^\mu a\right)-\frac{1}{2}m_a^2a^2+\frac{1}{\Lambda}\partial^\mu a\sum_{F}\bar{\psi}_F\mathbf{C}_F\gamma_\mu\psi_F\\
		&+g_s^2C_{GG}\frac{a}{\Lambda}G_{\mu\nu}^AG^{\mu\nu A}+g^2C_{WW}\frac{a}{\Lambda}W_{\mu\nu}^AW^{\mu\nu A}+g'^2C_{BB}\frac{a}{\Lambda}B_\mu B^\mu
	\end{split}
\end{align}
where $G_{\mu\nu}$, $W_{\mu\nu}$, and $B_\mu$ are the gauge fields for $SU(3)_C$, $SU(2)_L$, and $U(1)_Y$ respectively and $g_s$, $g$, and $g'$ are the corresponding coupling constants. The coefficients $\mathbf{C}_F$ are hermitian matrices and the coefficients $C_{ii}$ are the Wilson coefficients. Additional coupling terms to the Higgs boson can arise at dimension-6 order and higher~\cite{alp_colliders} as
\begin{equation}
	\label{eq:alp_6d_lagrangian}
	\mathcal{L}_\text{eff}^{D\geq6}=\frac{C_{ah}}{\Lambda^2}(\partial_\mu a)(\partial^\mu a)\phi^\dagger\phi+\frac{C'_{ah}}{\Lambda^2}m_a^2a^2\phi^\dagger\phi+\frac{C_{Zh}^{(7)}}{\Lambda^3}(\partial^\mu a)(\phi^\dagger iD_\mu\phi+\text{h.c.})\phi^\dagger\phi+...
\end{equation}
where the first two terms allow the ALP to decay to a pair of Higgs bosons and h.c. denotes the Hermitian conjugate. After electroweak symmetry breaking, the physical coupling terms of the ALPs to gauge bosons become
\begin{equation}
	\mathcal{L}_\text{eff}\ni e^2C_{\gamma\gamma}\frac{a}{\Lambda}F_{\mu\nu}F^{\mu\nu}+\frac{2e^2}{\sin\theta_w\cos\theta_w}C_{\gamma Z}\frac{a}{\Lambda}F_{\mu\nu}Z^{\mu\nu}+\frac{e^2}{\sin^2\theta_w\cos^2\theta_w}C_{ZZ}\frac{a}{\Lambda}Z_{\mu\nu}Z^{\mu\nu}
\end{equation}
where coefficients of the gauge fields are absorbed into the $C_{ii}$. Assuming the first term, which sets the decay rate of the ALP to photons, is the dominant decay mode, the lifetime of the ALP can be given by~\cite{Draper_2012}
\begin{equation}
	\gamma c\tau\simeq\left(\frac{1.15\unit{cm}}{16\left|C_{\gamma\gamma}\right|^2}\right)\left(\frac{\Lambda}{10\GeV}\right)^2\left(\frac{40\GeV}{m_a}\right)^4
\end{equation}
For suitable values of $C_{\gamma\gamma}$ and $\Lambda$, the ALPs can have ideal lifetimes and masses to probe at the LHC. These ALPs emerge naturally from several well motivated extensions to the standard model such as string theory~\cite{axion_st1,axion_st2,axion_st3} or next-to-minimal supersymmetry~\cite{axion_ss1,axion_ss2}.

Several models that predict ALPs can be shown to offer well motivated solutions to the issues with the Standard Model outlined in section~\ref{sec:theory_motivation}. Models such as the next-to-minimalist supersymmetry, Folded SUSY, Twin Higgs, or Quirky Little Higgs model include gauge singlets and offer solutions to the naturalness problem described in section~\ref{sec:hierarchy}~\cite{hierarchy,hierarchy2}. Coupling terms of the ALP to the Higgs boson as shown in equation~\ref{eq:alp_6d_lagrangian} are referred to as the Higgs portal, named such because the Higgs boson represents a portal to potential dark sector particles~\cite{Curtin_2014}. These dark sector particles, particularly the ALP, are promising candidates for dark matter~\cite{alp_dm1,alp_dm2,alp_dm3,alp_dm4,alp_dm5,alp_dm6}. Singlet scalar extensions can also be shown to provide the electroweak phase transition needed for EWBG, which would explain the baryon asymmetry of the universe described in section~\ref{sec:baryon_asymmetry}~\cite{alp_ewbg1,alp_ewbg2}. Additionally, if an ALP exists, it would contribute quantum corrections to the muon anomalous magnetic moment shown in figure~\ref{fig:muon_magnetic_moment_alp} and could explain the discrepancy between theory and experiment~\cite{alp_colliders}.
\begin{figure}[htb!]
	\centering
	\begingroup
	\tikzset{every picture/.style={scale=0.5}}
	% !TEX = root = ../../thesis.tex
\begin{tikzpicture}
	\begin{feynman}
		\coordinate[label=left:$\PGm$] (mu1) at (0,0);
		\coordinate[label=right:$\PGm$] (mu2) at (8,0);
		\coordinate (v1) at (2,1);
		\coordinate (v2) at (6,1);
		\coordinate (v3) at (4,{1+2*sqrt(3)});
		\coordinate[label=above:$\PGg$] (g1) at (4,6);
		\draw[fermion] (mu1) -- (v1);
		\draw[fermion] (v1) -- (v3);
		\draw[fermion] (v3) -- (v2);
		\draw[fermion] (v2) -- (mu2);
		\draw[photon] (v3) -- (g1);
		\draw[scalar] (v1) -- (v2);
		
		\node[label={left:$\PGm$}] at (3,{1+sqrt(3)}){};
		\node[label={below:$a$}] at (4, 1){};
		\node[label={right:$\PGm$}] at (5, {1+sqrt(3)}){};
	\end{feynman}
\end{tikzpicture}
	% !TEX = root = ../../thesis.tex
\begin{tikzpicture}
	\begin{feynman}
		\coordinate[label=left:$\PGm$] (mu1) at (0,0);
		\coordinate[label=right:$\PGm$] (mu2) at (8,0);
		\coordinate (v1) at (2,1);
		\coordinate (v2) at (6,1);
		\coordinate (v3) at (4,{1+2*sqrt(3)});
		\coordinate[label=above:$\PGg$] (g1) at (4,6);
		\draw[fermion] (mu1) -- (v1);
		\draw[boson] (v1) -- (v3);
		\draw[scalar] (v3) -- (v2);
		\draw[fermion] (v2) -- (mu2);
		\draw[photon] (v3) -- (g1);
		\draw[fermion] (v1) -- (v2);
		
		\node[label={left:$\PZ/\PGg$}] at (3,{1+sqrt(3)}){};
		\node[label={below:$\PGm$}] at (4, 1){};
		\node[label={right:$a$}] at (5, {1+sqrt(3)}){};
	\end{feynman}
\end{tikzpicture}
	\endgroup
	\caption[First order ALP contributions to the muon anomalous magnetic moment.]{First order ALP contributions to the muon anomalous magnetic moment.}
	\label{fig:muon_magnetic_moment_alp}
\end{figure}

It has been shown that several theories that predict ALPs offer resolutions to long standing problems with the Standard Model. The observation of long lived ALPs with masses on the electroweak scale decaying to photons would offer strong evidence of BSM physics due to the lack of Standard Model processes that could mimic this signal~\cite{Curtin_2014}. Particles such as the ALP serve as the motivation for the analysis presented in chapter~\ref{chap:ana}.