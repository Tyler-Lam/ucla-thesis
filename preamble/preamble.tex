% Packages
%\usepackage[T1]{fontenc} % Allows for 8-bit font encoding
\usepackage{setspace} % Sets space between lines
\usepackage{animate,pdfpages} % Allow for animated figures, importing PDF figures, and converting EPS to PDF
\usepackage[space]{grffile} % Extended file name support for graphics
\usepackage{float} % Improved interface for floating objects
\usepackage{subcaption} % Allows for subfigures within figures
\usepackage{caption} % Captions in floating environments
\usepackage{array} % Extending the array and tabular environments
\usepackage{multirow} % Create tabular cells spanning multiple rows
\usepackage{gensymb} % Generic symbols for both text and math mode
\usepackage{amsmath,amssymb,amsfonts,amsthm,mathrsfs,mathtools,accents} % Various packages for math typesetting
\usepackage{adjustbox,relsize} % Set font size relative to the current font size
\usepackage[notrig]{physics} % Physics symbols and units
\usepackage{siunitx} % Physics units
\usepackage{cancel,slashed} % Place lines through math and slashes through characters
\usepackage{xparse} % Generic document command parser
\usepackage{tikz,tikz-3dplot,circuitikz,pgfplots,tikz-cd,tikz-feynman} % Drawing environments
\usepackage{centernot} % Centered \not command
\usepackage{xcolor} % Color extensions for LaTeX and pdfLaTeX
\usepackage{listings} % Typeset text in WYSIWYG style and source code listings
\usepackage{fancyvrb} % Sophisticated verbatim text
\usepackage{lastpage} % Reference last page for Page N of M type footers
\usepackage[percent]{overpic} % Combine LaTeX commands over included graphics
\usepackage{empheq} % Emphasizing equations
\usepackage{comment} % Allows for commenting out large blocks of code
\usepackage[title]{appendix} % Customization of appendices
\usepackage{simpler-wick} % Allows for Wick contractions
\usepackage[htt]{hyphenat} % Enables hyphenation in \texttt environments
\usepackage{url} % Allows for \path command for long file paths
\usepackage[hidelinks,colorlinks=false]{hyperref} % Allow for hyperlinks
\usepackage{bookmark} % Automatically create bookmarks
\usepackage{xspace} % Allows for \xspace command to check if a space is needed
\usepackage{cite} % Makes citations with multiple references look better

% Configure TikZ
% Load TikZ libraries
\usetikzlibrary{patterns,plotmarks,arrows,decorations.pathmorphing,decorations.markings,shapes.geometric,calc,shapes.misc,3d}

\tikzset{>=latex}
\tikzset{pics/particle/.style n args={6}{code={
			\draw[rounded corners,fill=#1] (-1,-1) rectangle (1,1);
			\node at (0,0) [scale=2,anchor=mid] {#2};
			\node at (0,-0.8) [scale=0.8,anchor=mid] {#3};
			\node at (-1,0.8) [scale=0.5,anchor=mid west] {#4};
			\node at (-1,0.5) [scale=0.5,anchor=mid west] {#5};
			\node at (-1,0.2) [scale=0.5,anchor=mid west] {#6};
}}}

% Commands
\def\eqheight{-\the\dimexpr\fontdimen22\textfont2\relax}

% Configure listings
% Command for setting number of lines for code
\newlength{\numwidth}
\makeatletter
\newcommand{\setlinenum}[1]
{
	\setlength{\numwidth}{\widthof{\footnotesize{\lst@numberstyle{#1}}}}
	\def\lst@PlaceNumber{%
		\makebox[\numwidth+1em][l]{%
			\makebox[\numwidth][r]{\footnotesize\lst@numberstyle{\thelstnumber}}
		}
	}
}
\makeatother
\setlinenum{9} % Set to 9 lines of code by default

% Default style
\lstdefinestyle{default}
{
	showstringspaces=false,
	breaklines=true,
	frame=lines,
	basicstyle=\footnotesize\ttfamily,
	numbers=none,
}

% Plain style
\lstdefinestyle{plain}
{
	showstringspaces=false,
	breaklines=true,
	basicstyle=\footnotesize\ttfamily,
	numbers=none,
}

% LaTeX style
\lstdefinestyle{latex}
{
	language=LaTeX,
	showstringspaces=false,
	breaklines=true,
	frame=lines,
	basicstyle=\footnotesize\ttfamily,
	numberstyle=\footnotesize\ttfamily,
}

% Python style
\lstdefinestyle{python}
{
	language=Python,
	showstringspaces=false,
	breaklines=true,
	frame=lines,
	basicstyle=\footnotesize\ttfamily,
	numberstyle=\footnotesize\ttfamily,
}

% C++ style
\lstdefinestyle{cpp}
{
	language=C++,
	showstringspaces=false,
	breaklines=true,
	frame=lines,
	basicstyle=\footnotesize\ttfamily,
	numberstyle=\footnotesize\ttfamily,
}


% Configure captions
\captionsetup[table]{font={stretch=1.5}}
\captionsetup[figure]{font={stretch=1.5}}

% Configure itemize layers
\renewcommand{\labelitemii}{\ensuremath{\circ}}

% Enable bold math in section titles
\makeatletter
\g@addto@macro\bfseries{\boldmath}
\makeatother

% Commands
\renewcommand\qedsymbol{$\blacksquare$} % Change QED symbol to solid black box
\newcommand\numberthis{\addtocounter{equation}{1}\tag{\theequation}} % Number equations in align* environments

\sisetup{inter-unit-product=\ensuremath{{}\cdot{}}} % Change symbol for products of physical units
\renewcommand{\unit}{\ \si} % Define physical unit command with proper spacing
\DeclareSIUnit\clight{\text{\ensuremath{c}}} % Redefine speed of light so that it has no subscript

% Redefine command for unit vectors to add space
\let\oldvu\vu
\makeatletter
\renewcommand{\vu}{\@ifstar{\mathop{}\!\oldvu*}\mathop{}\!\oldvu}
\makeatother

% Create spacing for equals sign in align environments
\newcommand{\equad}{\mathrel{\phantom{=}}}

% Alphabet for blackboard bold numbers
\newcommand{\bbfamily}{\fontencoding{U}\fontfamily{bbold}\selectfont}
\DeclareMathAlphabet{\mathbbold}{U}{bbold}{m}{n}

\newcommand{\lumi}{\mathcal{L}} % Luminosity
\newcommand{\intlumi}{\mathcal{L}_\mathrm{int}} % Integrated luminosity
\newcommand{\lxy}{\ensuremath{L_{xy}}\xspace}
\newcommand{\pt}{\ensuremath{p_\mathrm{T}}\xspace} % Transverse momentum
\newcommand{\et}{\ensuremath{E_\mathrm{T}}\xspace} % Transverse energy
\newcommand{\VZ}{\ensuremath{Z}\xspace}
\newcommand{\VW}{\ensuremath{W}\xspace}
\newcommand{\VH}{\ensuremath{H}\xspace}
\newcommand{\ZH}{\ensuremath{ZH}\xspace}
\newcommand{\WH}{\ensuremath{WH}\xspace}
\newcommand{\ggZH}{\ensuremath{ggZH}\xspace}
\newcommand{\mphi}{\ensuremath{\text{m}_\Phi}\xspace} % Phi mass
\newcommand{\mgg}{\ensuremath{\text{m}_{\gamma\gamma}}\xspace} %m(gamma,gamma)
\newcommand{\ptgg}{\ensuremath{p_\mathrm{T}(\gamma\gamma)}\xspace} %pt(gamma,gamma)
\newcommand{\mll}{\ensuremath{m_{\ell\ell}}\xspace} % diphoton mass
\newcommand{\mh}{\ensuremath{m_{\mathrm{H}}}\space} %Higgs mass
